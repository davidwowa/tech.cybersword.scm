\documentclass[tikz,border=0pt]{standalone}
\usepackage[utf8]{inputenc}
\usepackage{csquotes}
\usepackage{xcolor}
\usepackage{graphicx}
\usepackage{pgffor}
\usepackage{listings}
\usepackage{array}
\usepackage{fontawesome}
\usepackage{amsmath}

\lstset{
    basicstyle=\ttfamily\fontsize{6}{8}\selectfont,
    breaklines=true,
    % backgroundcolor=\color{black},
    keywordstyle=\color{pink},
    commentstyle=\color{blue},
    stringstyle=\color{white},
    showstringspaces=false,
    frame=none,
    xleftmargin=0.6cm,
    xrightmargin=0.6cm
}

\begin{document}
\begin{tikzpicture}
\useasboundingbox (0,0) rectangle (10.8,10.8);

% Hintergrund in Schwarz
\fill[black] (0,0) rectangle (10.8,10.8);

% Zufällige Einsen und Nullen verteilen
\foreach \i in {1,...,5000} {
    \node[text=green, opacity=0.1, font=\ttfamily\fontsize{5}{6}\selectfont] at (rand*10.8, rand*10.8) {\pgfmathtruncatemacro{\random}{round(rand)}\random};
}

% \fill[red, opacity=0.1] (0.05,5.95) rectangle (10.75,10.75);
% \draw[red, thin] (0.05,5.95) rectangle (10.75,10.75); % 45% Höhe
\node[red, anchor=north west, font=\ttfamily\bfseries\fontsize{8}{9}\selectfont] at (0.1,10.65) {NVD DB Update {\Large{ CVE-2024-39556} - \textbf{Score:}{\Large{ 6.4 }}}};
% % ------------------------------------------------------------------------------------------------------------------------------
% \node[red, anchor=north west, font=\ttfamily\fontsize{8}{9}\selectfont, text width=10.6cm, align=center] at (0.1,10.25) {
% \newline
% \newline
% \newline
% If you want to succeed in penetration testing and cybersecurity, learn at least:
% \newline
% };
% ------------------------------------------------------------------------------------------------------------------------------
\fill[green, opacity=0.1] (0.05,1.15) rectangle (10.75,10.75);
\draw[green, thin] (0.05,1.15) rectangle (10.75,10.75); % 45% Höhe
% \node[green, anchor=north west, font=\ttfamily\bfseries\fontsize{8}{9}\selectfont] at (0.1,5.65) {Solution:};
% ------------------------------------------------------------------------------------------------------------------------------
\node[green, anchor=north west, font=\ttfamily\fontsize{8}{9}\selectfont, text width=10.6cm, align=center] at (0.1,10.25) {
\textbf{Components:}
\begin{scriptsize}
\begin{lstlisting}
 juniper junos 22.4
 juniper junos 23.4
 juniper junos *
 juniper junos 22.2
 juniper junos 21.4
 juniper junos 22.3
 juniper junos 23.2
 juniper junos 22.1
 juniper junos os evolved 22.2
 juniper junos os evolved *
 juniper junos os evolved 22.1
 juniper junos os evolved 22.4
 juniper junos os evolved 21.4
 juniper junos os evolved 22.3
 juniper junos os evolved 23.2
 juniper junos os evolved 23.4
\end{lstlisting}
\end{scriptsize}
\textbf{Description:}
\begin{scriptsize}
\begin{lstlisting}
A Stack-Based Buffer Overflow vulnerability in Juniper Networks Junos OS and Juniper Networks Junos OS Evolved may allow a local, low-privileged attacker with access to the CLI the ability to load a malicious certificate file, leading to a limited Denial of Service (DoS) or privileged code execution. By exploiting the 'set security certificates' command with a crafted certificate file, a malicious attacker with access to the CLI could cause a crash of the command management daemon (mgd), limited to the local user's command interpreter, or potentially trigger a stack-based buffer overflow. This issue affects:  Junos OS: * All versions before 21.4R3-S7, * from 22.1 before 22.1R3-S6, * from 22.2 before 22.2R3-S4, * from 22.3 before 22.3R3-S3, * from 22.4 before 22.4R3-S2, * from 23.2 before 23.2R2, * from 23.4 before 23.4R1-S1, 23.4R2;  Junos OS Evolved: * All versions before 21.4R3-S7-EVO, * from 22.1-EVO before 22.1R3-S6-EVO, * from 22.2-EVO before 22.2R3-S4-EVO, * from 22.3-EVO before 22.3R3-S3-EVO, * from 22.4-EVO before 22.4R3-S2-EVO, * from 23.2-EVO before 23.2R2-EVO, * from 23.4-EVO before 23.4R1-S1-EVO, 23.4R2-EVO.
\end{lstlisting}
\end{scriptsize}
\textbf{Main problems:}
\begin{scriptsize}
\begin{lstlisting}
121 no CWE Description available

\end{lstlisting}
\end{scriptsize}
};
% ------------------------------------------------------------------------------------------------------------------------------
\draw[green, thin] (0.05,0.05) rectangle (10.75,1.05); % 10% Höhe
% \node[green, anchor=north west, font=\ttfamily\bfseries\fontsize{5}{6}\selectfont] at (0.1,0.95) {Contact:};

% Tabelle 2x2 im Contact Block
\node[green, anchor=north west, font=\ttfamily\fontsize{8}{9}\selectfont, text width=10.6cm] at (0.1,0.95) {
\begin{tabular}{@{}p{4.8cm}@{}p{5cm}@{}}
\faGlobe\ https://cybersword.tech & \faLinkedin\ www.linkedin.com/in/cybersword-tech \\
\faInstagram\ ph1sher & \faTwitter\ @davidwowa \\
\end{tabular}
};
\end{tikzpicture}
\end{document}